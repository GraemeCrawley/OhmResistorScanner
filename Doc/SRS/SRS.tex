\documentclass{article}

\usepackage{booktabs}
\usepackage{tabularx}

\title{SE 3XA3: SRS\\ohm Resistor Scanner}

\author{Team \# 4, ohm
		\\ Ryan Marks (marksr2)
		\\ Graeme Crawley (crawleg)
		\\ Jonathan Brown (brownjs2)
}


\date{}

%% Comments

\usepackage{color}

\newif\ifcomments\commentstrue

\ifcomments
\newcommand{\authornote}[3]{\textcolor{#1}{[#3 ---#2]}}
\newcommand{\todo}[1]{\textcolor{red}{[TODO: #1]}}
\else
\newcommand{\authornote}[3]{}
\newcommand{\todo}[1]{}
\fi

\newcommand{\wss}[1]{\authornote{blue}{SS}{#1}}
\newcommand{\ds}[1]{\authornote{red}{DS}{#1}}
\newcommand{\mj}[1]{\authornote{red}{MSN}{#1}}
\newcommand{\cm}[1]{\authornote{red}{CM}{#1}}
\newcommand{\mh}[1]{\authornote{red}{MH}{#1}}

% team members should be added for each team, like the following
% all comments left by the TAs or the instructor should be addressed
% by a corresponding comment from the Team

\newcommand{\tm}[1]{\authornote{magenta}{Team}{#1}}


\begin{document}

\begin{table}[hp]
\caption{Revision History} \label{TblRevisionHistory}
\begin{tabularx}{\textwidth}{llX}
\toprule
\textbf{Date} & \textbf{Developer(s)} & \textbf{Change}\\
\midrule
2016-09-26 & Ryan & Created template with section headings\\
... & ... & ...\\
\bottomrule
\end{tabularx}
\end{table}

\newpage

\maketitle



\section{Project Drivers}

\subsection{The Purpose of the Project }
\subsection{The Stakeholders}


\section{Project Constraints}

\subsection{Mandated Constraints}
\subsection{Naming Conventions and Terminology }
\subsection{Relevant Facts and Assumptions}


\section{Functional Requirements}

\subsection{The Scope of the Work}
\subsection{Business Data Model \& Data Dictionary}
\subsection{The Scope of the Product}
\subsection{Functional Requirements}


\section{Non-functional Requirements}

\subsection{Look and Feel Requirements }
\begin{enumerate}
\item The application shall have a very minimal user interface focussed primarily around the view of the camera.
\end{enumerate}
\subsection{Usability and Humanity Requirements}
\begin{enumerate}
\item The application shall be easily usable by people aged 8 to 70.
\item The application shall be suitable for a user with a minimal understanding of resistor colour codes and very little training.
\item The application shall not require the understanding of any particular language as all options symbolically.
\item The application shall be usable by those with colour blindness.
\end{enumerate}
\subsection{Performance Requirements}
\subsubsection{Speed and Latency Requirements}
\begin{enumerate}
\item The application should launch in no more than 5 seconds.
\item The algorithm used to process the income stream of images should be able to process between 15 and 30 images per second on a standard mobile phone.
\end{enumerate}
\subsubsection{Precision or Accuracy Requirements}
\begin{enumerate}
\item The application should identify an input as the correct class of resistor 95\% of the time.
\end{enumerate}
\subsubsection{Reliability and Availability Requirements}
\begin{enumerate}
\item The application will not require internet access.
\end{enumerate}
\subsection{Operational and Environmental Requirements}
\begin{enumerate}
\item
\end{enumerate}
\subsection{Maintainability and Support Requirements}
\begin{enumerate}
\item
\end{enumerate}
\subsection{Security Requirements}
\begin{enumerate}
\item
\end{enumerate}
\subsection{Cultural Requirements}
\begin{enumerate}
\item
\end{enumerate}
\subsection{Legal Requirements}
\begin{enumerate}
\item
\end{enumerate}

\section{Project Issues}

\subsection{Open Issues }
\subsection{Off-the-Shelf Solutions}
\subsection{New Problems}
\subsection{Tasks}
\subsection{Migration to the New Product}
\subsection{Risks}
\subsection{Costs}
\subsection{User Documentation and Training}
\subsection{Waiting Room}
\subsection{Ideas for Solutions}


\end{document}