\documentclass{article}

\usepackage{tabularx}
\usepackage{booktabs}

\title{SE 3XA3: Problem Statement\\Title of Project}

\author{Team \# 4, Team Name
		\\ Ryan Marks (marksr2)
		\\ Graeme Crawley (crawleg)
		\\ Jonathan Brown (brownjs2)
}

\date{}

\input{../Comments}

\begin{document}

\begin{table}[hp]
\caption{Revision History} \label{TblRevisionHistory}
\begin{tabularx}{\textwidth}{llX}
\toprule
\textbf{Date} & \textbf{Developer(s)} & \textbf{Change}\\
\midrule
2016-09-23 & Ryan, Graeme, Jonathan & Contents of a collaboratively produced Google Doc moved into Tex\\
\bottomrule
\end{tabularx}
\end{table}

\newpage

\maketitle

\section{Statement of problem to be solved}

Many people around the world use electrical circuits, which require resistors.
These resistors have color codes that allow for the understanding of resistance levels of each resistor.
We aim to make it easier to read these resistor values by creating an application that can label a resistor using a camera, 
allowing the user to simply point the camera and understand the value of that resistor. 
The value of this project is in the convenience it could provide to everyday hobbyists, along with the accessibility it could provide to people who are color blind.

\section{Stakeholders}

The main stakeholders for this application are the development team and our eventual users. 
However there are a number of different use cases for this application, so we will divide our users into the broad demographics of electronics beginners, colorblind hobbyists, and power users. Beginners are people new to electronics who don’t know how to read resistor color codes and don’t yet own a resistance meter to allow for measuring resistance rather than reading it.
Another group that has difficulty reading resistor color codes are colorblind hobbyists. 
This particular set of people would benefit greatly as without this application they would have to rely on the word from those who are not color blind, or from the matching of shades. 
Finally power users who want to quickly find a resistor without manually searching, possibly with other use cases.

\section{Contextual Environment}

We are targeting the hobbyist electronics market in which axial package resistors with color band identification is extremely common. 
In a professional context resistors are mass packaged by resistance making identification trivial. 
This contrasts greatly with hobbyist electronics where parts are stored with greatly varying levels of organization, sometimes as poor as a single drawer of loose components of all kinds.
Electronics hobbyists must identify components in a wide variety of working environments. 
Most relevant to our problem is the visual environment as there exists a wide variety of surfaces against which resistors must be identified.


\end{document}