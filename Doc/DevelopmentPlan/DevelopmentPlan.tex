\documentclass{article}

\usepackage{booktabs}
\usepackage{tabularx}

\title{SE 3XA3: Development Plan \\ohm}

\author{Team 4, ohm
		\\Jonathan Brown, brownjs2
		\\Graeme Crawley, crawleg
		\\Ryan Marks, marksr2
}

\date{}

\input{../Comments}

\begin{document}

\begin{table}[hp]
\caption{Revision History} \label{TblRevisionHistory}
\begin{tabularx}{\textwidth}{llX}
\toprule
\textbf{Date} & \textbf{Developer(s)} & \textbf{Change}\\
\midrule
Date1 & Name(s) & Description of changes\\
Date2 & Name(s) & Description of changes\\
... & ... & ...\\
\bottomrule
\end{tabularx}
\end{table}

\newpage

\maketitle

Put your introductory blurb here.

\section{Team Meeting Plan}

\section{Team Communication Plan}

\section{Team Member Roles}

\section{Git Workflow Plan}

We will follow a feature branching model for source code and a linear commit process for document changes.
In addition we will mandate code review and merge requests to force understanding of all code.
Merge requests should be "Thumbs Upped" on Gitlab by both other developers before being merged by the author however this is not a hard rule:
If a developer feels they must commit code to master or merge without approval they can, so long as they can feel the circumstances are justifiable.
This was chosen over the gitflow model as gitflow's benefits are for tracking a software system with regular releases that must be maintained.
As our software is in such early stages that we will not have legacy releases to maintain, the benefits of gitflow are negligible when compared to the organizational costs.
The documents however will use a linear commiting style to master as documents are easier to merge and there is a good deal of value in everyone being aware of unfinished sections of the document.

\section{Proof of Concept Demonstration Plan}

\section{Technology}

\section{Coding Style}

\section{Project Schedule}

Provide a pointer to your Gantt Chart.

\section{Project Review}

\end{document}