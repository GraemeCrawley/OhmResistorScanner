\documentclass[12pt, titlepage]{article}
\usepackage{booktabs}
\usepackage{tabularx}
\usepackage{hyperref}
\hypersetup{
    colorlinks,
    citecolor=black,
    filecolor=black,
    linkcolor=red,
    urlcolor=blue
}
\usepackage[round]{natbib}
\title{SE 3XA3: Test Plan\\ Ohm: Resistor Scanner}
\author{Team 4, ohm
		\\ Jonathan Brown, brownjs2
		\\ Graeme Crawley, crawleg
		\\ Ryan Marks, marksr2
}
\date{\today}
%% Comments

\usepackage{color}

\newif\ifcomments\commentstrue

\ifcomments
\newcommand{\authornote}[3]{\textcolor{#1}{[#3 ---#2]}}
\newcommand{\todo}[1]{\textcolor{red}{[TODO: #1]}}
\else
\newcommand{\authornote}[3]{}
\newcommand{\todo}[1]{}
\fi

\newcommand{\wss}[1]{\authornote{blue}{SS}{#1}}
\newcommand{\ds}[1]{\authornote{red}{DS}{#1}}
\newcommand{\mj}[1]{\authornote{red}{MSN}{#1}}
\newcommand{\cm}[1]{\authornote{red}{CM}{#1}}
\newcommand{\mh}[1]{\authornote{red}{MH}{#1}}

% team members should be added for each team, like the following
% all comments left by the TAs or the instructor should be addressed
% by a corresponding comment from the Team

\newcommand{\tm}[1]{\authornote{magenta}{Team}{#1}}

\begin{document}
\maketitle
\pagenumbering{roman}
\tableofcontents
\listoftables
\listoffigures
\begin{table}[bp]
\caption{\bf Revision History}
\begin{tabularx}{\textwidth}{p{3cm}p{2cm}X}
\toprule {\bf Date} & {\bf Version} & {\bf Notes}\\
\midrule
Date 1 & 1.0 & Notes\\
Date 2 & 1.1 & Notes\\
\bottomrule
\end{tabularx}
\end{table}
\newpage
\pagenumbering{arabic}
\section{General Information}
\subsection{Purpose}
\par This document will describe the testing procedure used to ensure the correct functionality of Group 4's 3XA3 Project, Ohm. While the implementaion of the project is not complete, it is important to have plan tests that verify that the project complies with the specifications and requirements set out in the SRS. These tests are necessary in order to produce a high quality end product, as well as track and manage the progress of the group.
\subsection{Scope}
\par The tests prescribed in the test plan should verify the efficacy of the resistor band detection, the colour selection and the resistor body detection (note: the resistor body detection has not yet been implemented).
\subsection{Acronyms, Abbreviations, and Symbols}
	
\begin{table}[hbp]
\caption{\textbf{Table of Abbreviations}} \label{Table}
\begin{tabularx}{\textwidth}{p{3cm}X}
\toprule
\textbf{Abbreviation} & \textbf{Definition} \\
\midrule
Abbreviation1 & Definition1\\
Abbreviation2 & Definition2\\
\bottomrule
\end{tabularx}
\end{table}
\begin{table}[!htbp]
\caption{\textbf{Table of Definitions}} \label{Table}
\begin{tabularx}{\textwidth}{p{3cm}X}
\toprule
\textbf{Term} & \textbf{Definition}\\
\midrule
Term1 & Definition1\\
Term2 & Definition2\\
\bottomrule
\end{tabularx}
\end{table}	

\subsection{Overview of Document}
The Test Plan will contain five main sections excluding this introductory one, as well as an appendix. 

\section{Plan}
	
\subsection{Software Description}
Ohm will allow users with a desktop computer or smartphone equipped camera to determine the resistance of a standard 4-band resistor by placing it within the camera frame. The software will, using Opencv, detect and read the colour bands of the resistor automatically to determine it's resistance. This software will be implemented in Java.
\subsection{Test Team}
All members of Group 4, Jonathan Brown, Graeme Crawley, and Ryan Marks will be responsible for testing components of the application as well as the application as a whole.
\subsection{Automated Testing Approach}

\subsection{Testing Tools}
\par JUnit 4 will be the tool for testing individual modules of the software.
\subsection{Testing Schedule}

INTSERT TABLE HERE
		
See Gantt Chart at the following url ...
\section{System Test Description}
	
\subsection{Tests for Functional Requirements}
\subsection{Tests for Functional Requirements}
\subsubsection{User Input}
		
\paragraph{Resistor Scanning}
\begin{enumerate}
\item{test-id1\\}
Type: Functional, Dynamic, Manual, Static etc.
					
Initial State: 
App is open, camera is on
					
Input: 
Camera is pointed at resistor in incorrect orientation
					
Output: 
No output
					
How test will be performed: 
The user will point the camera at a resistor without lining up the red line on the app to intersect the four bands of the resistor.
					
\item{test-id2\\}
Type: Functional, Dynamic, Manual, Static etc.
					
Initial State: 
App is open, camera is on
					
Input: 
Camera is pointed at resistor in correct orientation with red line crossing through first stripe from either side
					
Output: 
No output

\item{test-id3\\}
Type: Functional, Dynamic, Manual, Static etc.
					
Initial State: 
App is open, camera is on
					
Input: 
Camera is pointed at resistor in correct orientation with red line crossing through two stripes from either side
					
Output: 
No output	

\item{test-id4\\}
Type: Functional, Dynamic, Manual, Static etc.
					
Initial State: 
App is open, camera is on
					
Input: 
Camera is pointed at resistor in correct orientation with red line crossing through three stripes from either side
					
Output: 
No output	

\item{test-id5\\}
Type: Functional, Dynamic, Manual, Static etc.
					
Initial State: 
App is open, camera is on
					
Input: 
Camera is pointed at resistor in correct orientation with red line crossing through four stripes from either side
					
Output: 
Resistance value is displayed on the screen	
					
How test will be performed: 
\end{enumerate}
\subsubsection{Area of Testing2}
...
\subsection{Tests for Nonfunctional Requirements}
\subsubsection{Area of Testing1}
		
\paragraph{Title for Test}
\begin{enumerate}
\item{test-id1\\}
Type: 
					
Initial State: 
					
Input/Condition: 
					
Output/Result: 
					
How test will be performed: 
					
\item{test-id2\\}
Type: Functional, Dynamic, Manual, Static etc.
					
Initial State: 
					
Input: 
					
Output: 
					
How test will be performed: 
\end{enumerate}
\subsubsection{Area of Testing2}
...
\subsection{Tests for Nonfunctional Requirements}
\subsubsection{Area of Testing1}
		
\paragraph{Title for Test}
\begin{enumerate}
\item{test-id1\\}
Type: 
					
Initial State: 
					
Input/Condition: 
					
Output/Result: 
					
How test will be performed: 
					
\item{test-id2\\}
Type: Functional, Dynamic, Manual, Static etc.
					
Initial State: 
					
Input: 
					
Output: 
					
How test will be performed: 
\end{enumerate}
\subsubsection{Area of Testing2}
...
\section{Tests for Proof of Concept}
\subsection{Band Identification}
		
\paragraph{Title for Test}
\begin{enumerate}
\item{PC-BI-1\\}
\\
Type: Functional, Dynamic, Manual, Static etc.
					
Initial State: Software not running
					
Input: Run Command
					
Output: Software displays static image of resistor with no bands selected.
					
How test will be performed: 
					
\item{PC-BI-2\\}
\\
Type: Functional, Dynamic, Manual, Static etc.
					
Initial State: Software running, no axis selected
					
Input: Select axis directly across resistor body.
					
Output:  A ring centered around
					
How test will be performed: 
\end{enumerate}
\subsection{Area of Testing2}
...
	
\section{Comparison to Existing Implementation}	
				
\section{Unit Testing Plan}	
\par JUnit4 will be used to perform unit testing for this application.	
\subsection{Unit testing of internal functions}
\par Unit testing will be performed on any independant functions where comparison to expected results is possible. This project features several prime candidates for such testing: expected band locations, and expected colour selection. Once this unit testing is performed these individual modules can be combined and automated dyamic and functional testing can be performed to verify the final resistance output. Modules featuring image processing (the majority of them) will be tested using a stub to pass them static images in place of camera input. The application behaves quite linearly, allowing for complete test coverage.
\subsection{Unit testing of output files}		
\par This project does not feature any output files.
\bibliographystyle{plainnat}
\bibliography{SRS}
\newpage
\section{Appendix}
This is where you can place additional information.
\subsection{Symbolic Parameters}
The definition of the test cases will call for SYMBOLIC\_CONSTANTS.
Their values are defined in this section for easy maintenance.
\subsection{Usability Survey Questions?}
This is a section that would be appropriate for some teams.
\end{document}