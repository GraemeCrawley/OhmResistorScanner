\documentclass[12pt, titlepage]{article}
\usepackage{booktabs}
\usepackage{tabularx}
\usepackage{hyperref}
\hypersetup{
    colorlinks,
    citecolor=black,
    filecolor=black,
    linkcolor=red,
    urlcolor=blue
}
\usepackage[round]{natbib}
\title{SE 3XA3: Test Plan\\ Ohm: Resistor Scanner}
\author{Team 4, ohm
		\\ Jonathan Brown, brownjs2
		\\ Graeme Crawley, crawleg
		\\ Ryan Marks, marksr2
}
\date{\today}
%% Comments

\usepackage{color}

\newif\ifcomments\commentstrue

\ifcomments
\newcommand{\authornote}[3]{\textcolor{#1}{[#3 ---#2]}}
\newcommand{\todo}[1]{\textcolor{red}{[TODO: #1]}}
\else
\newcommand{\authornote}[3]{}
\newcommand{\todo}[1]{}
\fi

\newcommand{\wss}[1]{\authornote{blue}{SS}{#1}}
\newcommand{\ds}[1]{\authornote{red}{DS}{#1}}
\newcommand{\mj}[1]{\authornote{red}{MSN}{#1}}
\newcommand{\cm}[1]{\authornote{red}{CM}{#1}}
\newcommand{\mh}[1]{\authornote{red}{MH}{#1}}

% team members should be added for each team, like the following
% all comments left by the TAs or the instructor should be addressed
% by a corresponding comment from the Team

\newcommand{\tm}[1]{\authornote{magenta}{Team}{#1}}

\begin{document}
\maketitle
\pagenumbering{roman}
\tableofcontents
\newpage
\listoftables
\begin{table}[h]
\caption{\bf Revision History}
\begin{tabularx}{\textwidth}{p{3cm}p{2cm}X}
\toprule {\bf Date} & {\bf Version} & {\bf Notes}\\
\midrule
October 31st 2016 & 0.0 & First Revision of Testing Plan\\
\bottomrule
\end{tabularx}
\end{table}
\newpage
\pagenumbering{arabic}
\section{General Information}
\subsection{Purpose}
\par This document will describe the testing procedure used to ensure the correct functionality of Group 4's 3XA3 Project, Ohm. While the implementaion of the project is not complete, it is important to have plan tests that verify that the project complies with the specifications and requirements set out in the SRS. These tests are necessary in order to produce a high quality end product, as well as track and manage the progress of the group.
\subsection{Scope}
\par The tests prescribed in the test plan should verify the efficacy of the resistor band detection, the colour selection and the resistor body detection (note: the resistor body detection has not yet been implemented).

%\subsection{Acronyms, Abbreviations, and Symbols}	
%\begin{table}[hbp]
%\caption{\textbf{Table of Abbreviations}} \label{Table}
%\begin{tabularx}{\textwidth}{p{3cm}X}
%\toprule
%\textbf{Abbreviation} & \textbf{Definition} \\
%\midrule
%Abbreviation1 & Definition1\\
%Abbreviation2 & Definition2\\
%\bottomrule
%\end{tabularx}
%\end{table}
%\begin{table}[!htbp]
%\caption{\textbf{Table of Definitions}} \label{Table}
%\begin{tabularx}{\textwidth}{p{3cm}X}
%\toprule
%\textbf{Term} & \textbf{Definition}\\
%\midrule
%Term1 & Definition1\\
%Term2 & Definition2\\
%\bottomrule
%\end{tabularx}
%\end{table}	

\subsection{Overview of Document}
The Test Plan will contain five main sections excluding this introductory one. An outline of the contents of the document can be found in the table of contents.

\section{Plan}
	
\subsection{Software Description}
Ohm will allow users with a desktop computer or smart phone equipped camera to determine the resistance of a standard 4-band resistor by placing it within the camera frame. The software will, using Opencv, detect and read the colour bands of the resistor automatically to determine it's resistance. This software will be implemented in Java.
\subsection{Test Team}
All members of Group 4, Jonathan Brown, Graeme Crawley, and Ryan Marks will be responsible for testing components of the application as well as the application as a whole.
\subsection{Automated Testing Approach}
Automated testing is inherently difficult with respect to this problem, as much of the testing conducted is required to be supervised and manual. What automated testing can be done will be completed during the unit testing phase and completed using JUnit.
\subsection{Testing Tools}
\par JUnit 4 will be the tool for unit testing of the software.
\subsection{Testing Schedule}
Please reference the Gantt chart in the project schedule folder of this repository.

\url{https://gitlab.cas.mcmaster.ca/marksr2/ohm/blob/master/ProjectSchedule/3XA3_Group_4_Gantt_Chart.gan}

\section{System Test Description}
	
\subsection{Tests for Functional Requirements}
\subsubsection{User Input}
\paragraph{Warning Box}
\begin{enumerate}
\item{F-FDM-1\\}
Type: Functional, Dynamic, Manual
					
Initial State: 
App is not yet launched. Phone is turned on with app installed and available for launch.
					
Input: 
App is launched
					
Output: 
Camera starts, warning box appears over camera view
					
How test will be performed: 
The user will launch the app which will automatically display a warning box upon opening.

\item{F-UDM-2\\}
Type: Unit, Dynamic, Manual
					
Initial State: 
Warning box is displayed on screen.
					
Input: 
"X" on the box is pressed
					
Output: 
Warning box closes, showing camera
					
How test will be performed: 
The user will press the "X" on the warning box, which will then close the warning box.

\item{F-UDM-3\\}
Type: Unit, Dynamic, Manual
					
Initial State: 
Warning box is displayed on screen.
					
Input: 
Screen is pressed anywhere that isn't the "X" on the warning box
					
Output: 
No output
					
How test will be performed: 
The user will press the space around "X" on the warning box, which will do nothing.

\end{enumerate}

\paragraph{Screen}
\begin{enumerate}
\item{F-UDM-4\\}
Type: Unit, Dynamic, Manual
					
Initial State: 
App is open with camera display live video to the screen. Red horizontal line is visible and camera feed is operational, indicating readiness to scan.
					
Input: 
Screen is tapped
					
Output: 
No output
					
How test will be performed: 
The user will  tap the screen while the app is running, resulting in no change

\end{enumerate}

\subsubsection{Resistor Scanning}
\begin{enumerate}
\item{F-FDM-5\\}
Type: Functional, Dynamic, Manual
					
Initial State: 
App is open with camera display live video to the screen. Red horizontal line is visible and camera feed is operational, indicating readiness to scan.
					
Input: 
Camera is pointed at resistor in incorrect orientation
					
Output: 
No output
					
How test will be performed: 
The user will point the camera at a resistor without lining up the red line on the app to intersect any of the four bands of the resistor.
					
\item{F-FDM-6\\}
Type: Functional, Dynamic, Manual
					
Initial State: 
App is open with camera display live video to the screen. Red horizontal line is visible and camera feed is operational, indicating readiness to scan.
					
Input: 
Camera is pointed at resistor in correct orientation with red line crossing through first stripe from either side
					
Output: 
No output

How test will be performed: 
The user will point the camera at a resistor with the red line intersecting one of the four bands of the resistor.

\item{F-FDM-7\\}
Type: Functional, Dynamic, Manual
					
Initial State: 
App is open with camera display live video to the screen. Red horizontal line is visible and camera feed is operational, indicating readiness to scan.
					
Input: 
Camera is pointed at resistor in correct orientation with red line crossing through two stripes from either side
					
Output: 
No output	

How test will be performed: 
The user will point the camera at a resistor with the red line intersecting two of the four bands of the resistor.

\item{F-FDM-8\\}
Type: Functional, Dynamic, Manual
					
Initial State: 
App is open with camera display live video to the screen. Red horizontal line is visible and camera feed is operational, indicating readiness to scan.
					
Input: 
Camera is pointed at resistor in correct orientation with red line crossing through three stripes from either side
					
Output: 
No output	

How test will be performed: 
The user will point the camera at a resistor with the red line intersecting three of the four bands of the resistor.

\item{F-FDM-9\\}
Type: Functional, Dynamic, Manual
					
Initial State: 
App is open with camera display live video to the screen. Red horizontal line is visible and camera feed is operational, indicating readiness to scan.
					
Input: 
Camera is pointed at resistor in correct orientation with red line crossing through four stripes from either side
					
Output: 
Resistance value is displayed on the screen	

How test will be performed: 
The user will point the camera at a resistor with the red line intersecting all four bands of the resistor.

\item{F-FDM-10\\}
Type: Functional, Dynamic, Manual
					
Initial State: 
App is open with camera display live video to the screen. Red horizontal line is visible and camera feed is operational, indicating readiness to scan.
					
Input: 
Camera is on with no resistor visible on screen
					
Output: 
No output

How test will be performed: 
The user will point the camera so that no resistor is visible on screen.

\end{enumerate}








\subsection{Tests for Nonfunctional Requirements}
\subsubsection{User Tests}
\begin{enumerate}
\item{NF-FDM-11\\}
Type: Functional, Dynamic, Manual
					
Initial State: 
App is open with camera display live video to the screen. Red horizontal line is visible and camera feed is operational, indicating readiness to scan.
					
Input/Condition: 
100 resistors scanned
					
Output/Result: 
95 resistors or more will be identified accurately
					
How test will be performed: 
100 resistors will be scanned consecutively by hand, their values being displayed on screen. The values will be recorded and compared to he manually calculated values of each resistor. At least 95 resistors will be identified accurately.
					
\item{NF-FDM-12\\}
Type: Functional, Dynamic, Manual
					
Initial State: 
App is open with camera display live video to the screen. Red horizontal line is visible and camera feed is operational, indicating readiness to scan.
					
Input: 
Give the app and a resistor to a user between the ages of 8 and 70
					
Output: 
The user will be able to identify the value of the resistor.
					
How test will be performed: 
The user will be given a resistor and the app. The app will be easy to use and allow the resistance to be measured.

\item{NF-FDM-13\\}
Type: Functional, Dynamic, Manual
					
Initial State: 
App is open with camera display live video to the screen. Red horizontal line is visible and camera feed is operational, indicating readiness to scan.
					
Input: 
Give the app to a user who doesn't have Internet connection
					
Output: 
The user will be able to identify the value of the resistor
					
How test will be performed: 
The user will be given a resistor and the app. The app will be easy to use and allow the resistance to be measured without the need for Internet.

\end{enumerate}
\subsubsection{Launch Tests}
\begin{enumerate}
\item{NF-FDM-14\\}
Type: Functional, Dynamic, Manual
					
Initial State: 
App is not yet launched. Phone is turned on with app installed and available for launch.
					
Input: 
App is launched
					
Output: 
The warning box will display text
					
How test will be performed: 
The app will be launched and display a warning box which contains text. The text will notify the user that the app should not be used in safety critical systems.

\item{NF-FDM-15\\}
Type: Functional, Dynamic, Manual
					
Initial State: 
App is not yet launched. Phone is turned on with app installed and available for launch.
					
Input: 
App is launched
					
Output: 
The app will display the warning box within 5 seconds
					
How test will be performed: 
The app be launched and display the warning box within 5 seconds


\end{enumerate}




\section{Tests for Proof of Concept}
\subsection{Band Identification}
\begin{enumerate}
\item{PC-BI-1\\}
\\
Type: Functional, Dynamic, Manual
					
Initial State: Software not yet running. Software ready to be executed.
					
Input: Run Command
					
Output: Software displays static image of resistor with no bands selected.
					
How test will be performed: Manual execution.
					
\item{PC-BI-2\\}
\\
Type: Functional, Dynamic, Manual.
					
Initial State: Software running with image displayed on screen for axis selection, no axis selected
					
Input: Select axis directly across resistor body.
					
Output:  A ring centered around each band of the appropriate colour.
					
How test will be performed:  Manual execution.

\item{PC-BI-3\\}
\\
Type: Functional, Dynamic, Manual
					
Initial State: Software running with image displayed on screen for axis selection, no axis selected
					
Input: Select axis incorrectly.
					
Output: Exception is thrown
					
How test will be performed: Manual execution.

\end{enumerate}

\section{Comparison to Existing Implementation}
There are two tests that compare the program to the Existing Implementation of the program. Please refer to:
\begin{itemize}
\item test F-FDM-9 in Resistor Scanning for Functional Requirements\\
\\
This test shows the core functionality of the application and its similarity to the existing implementation. Both applications used a red line to align the resistor to the center of the screen, and outputted the calculated resistance to the screen.
\\
\item test NF-FDM-15 in Launch Tests for Nonfunctional Requirements\\
\\
This test shows that the application performs similarly to the existing application in its speed and performance.
\end{itemize}

\section{Unit Testing Plan}	
\par JUnit4 will be used to perform unit testing for this application. Due to the lack of well defined correctness for the problem, the coverage of testing is not something that will be a focus. However, unit tests will be done for the Colour Mapping Module due to the well defined expected outcome for individual colours and the need to test a large collection of samples in a small period of time.
\subsection{Unit testing of internal functions}
\par Unit testing will be performed on any independent functions where comparison to expected results is possible. This project features several prime candidates for such testing: expected band locations, and expected colour selection. Once this unit testing is performed these individual modules can be combined and automated dynamic and functional testing can be performed to verify the final resistance output. Modules featuring image processing (the majority of them) will be tested using a stub to pass them static images in place of camera input.
\subsection{Unit testing of output files}		
\par This project does not feature any output files.
\bibliographystyle{plainnat}
\bibliography{SRS}
\end{document}