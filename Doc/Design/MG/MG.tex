\documentclass[12pt, titlepage]{article}

\usepackage{fullpage}
\usepackage[round]{natbib}
\usepackage{multirow}
\usepackage{booktabs}
\usepackage{tabularx}
\usepackage{graphicx}
\usepackage{float}
\usepackage{hyperref}
\hypersetup{
    colorlinks,
    citecolor=black,
    filecolor=black,
    linkcolor=red,
    urlcolor=blue
}
\usepackage[round]{natbib}

\newcounter{acnum}
\newcommand{\actheacnum}{AC\theacnum}
\newcommand{\acref}[1]{AC\ref{#1}}

\newcounter{ucnum}
\newcommand{\uctheucnum}{UC\theucnum}
\newcommand{\uref}[1]{UC\ref{#1}}

\newcounter{mnum}
\newcommand{\mthemnum}{M\themnum}
\newcommand{\mref}[1]{M\ref{#1}}

\title{SE 3XA3: Software Requirements Specification\\ ohm}

\author{Team 4, Ohm
		\\ Jonathan Brown, brownjs2
		\\ Graeme Crawley, crawleg
		\\ Ryan Marks, marksr2
}

\date{\today}

%% Comments

\usepackage{color}

\newif\ifcomments\commentstrue

\ifcomments
\newcommand{\authornote}[3]{\textcolor{#1}{[#3 ---#2]}}
\newcommand{\todo}[1]{\textcolor{red}{[TODO: #1]}}
\else
\newcommand{\authornote}[3]{}
\newcommand{\todo}[1]{}
\fi

\newcommand{\wss}[1]{\authornote{blue}{SS}{#1}}
\newcommand{\ds}[1]{\authornote{red}{DS}{#1}}
\newcommand{\mj}[1]{\authornote{red}{MSN}{#1}}
\newcommand{\cm}[1]{\authornote{red}{CM}{#1}}
\newcommand{\mh}[1]{\authornote{red}{MH}{#1}}

% team members should be added for each team, like the following
% all comments left by the TAs or the instructor should be addressed
% by a corresponding comment from the Team

\newcommand{\tm}[1]{\authornote{magenta}{Team}{#1}}


\begin{document}

\maketitle

\pagenumbering{roman}
\tableofcontents
\listoftables
\listoffigures

\begin{table}[bp]
\caption{\bf Revision History}
\begin{tabularx}{\textwidth}{p{3cm}p{2cm}X}
\toprule {\bf Date} & {\bf Version} & {\bf Notes}\\
\midrule
11/13/2016 & 1.0 & Initial Revision\\
\bottomrule
\end{tabularx}
\end{table}

\newpage

\pagenumbering{arabic}

\section{Introduction}
\subsection{Overview}
The resistance scanner project is the re-implementation of an open-source software application that allows a user to scan a resistor using their camera and read an on screen value representing the resistance of that resistor.

\subsection{Context}
\par This document is the Module Guide (MG). The Module Guide, written after the SRS, is used as an outline for the implementaion of all functional and nonfunctional specified in the SRS. This is done by the decomposition of the system into modules, showing it's structure and design. A brief explination of each module is given and the requirements outlined in the SRS are directly mapped to their modular implementations. An explination of the relationships between each module are also given. This is useful both for devlopers and maintainers as it allows them to more easily identify parts of the software.
\par For a more detailed description of the individual components of the system, the MIS is used to explain the semantics and syntax of the member functions of the classes used in the implementation of the modules.

\subsection{Design Principles}
This project implements the design principles of Single Responsibility, Separation of Concerns and Least Knowledge.
\begin{itemize}
\item \textbf{Single Responsibility:} 
\item \textbf{Separation of Concerns:}
\item \textbf{Principle of Least Knowledge:}
\end{itemize}

\section{Anticipated and Unlikely Changes} \label{SecChange}

\subsection{Anticipated Changes} \label{SecAchange}
\begin{description}
\item[\refstepcounter{acnum} \actheacnum \label{acInput}:] The method of retrieving input images (Camera/Static Image).
\item[\refstepcounter{acnum} \actheacnum \label{acAxis}:] The method of determining the axis of a resistor.
\item[\refstepcounter{acnum} \actheacnum \label{acPlatform}:] The platform of the application (Desktop to Mobile).
\end{description}

\subsection{Unlikely Changes} \label{SecUchange}
\begin{description}
\item[\refstepcounter{ucnum} \uctheucnum \label{ucIO}:] The format of the output data.
\item[\refstepcounter{ucnum} \uctheucnum \label{ucInput}:] The method used to scan the bands resistor.
\item[\refstepcounter{ucnum} \uctheucnum \label{ucInput}:] The method used to get colours from the bands of the resistor.
\item[\refstepcounter{ucnum} \uctheucnum \label{ucInput}:] The method used to calculate values from the resistor colours.
\item[\refstepcounter{ucnum} \uctheucnum \label{ucInput}:] The content of the output.
\end{description}

\section{Module Hierarchy} \label{SecMH}

This section provides an overview of the module design. Modules are summarized
in a hierarchy decomposed by secrets in Table \ref{TblMH}. The modules listed
below, which are leaves in the hierarchy tree, are the modules that will
actually be implemented.

\begin{table}[h]
\centering
\label{my-label}
\begin{tabular}{cl}
\hline
\multicolumn{1}{l}{\textbf{Module Number}} & \textbf{Module Name}    \\ \hline
\textbf{M1}                                & Camera Input Module     \\ \hline
\textbf{M2}                                & Image Input Module      \\ \hline
\textbf{M3}                                & User Interface Module   \\ \hline
\textbf{M4}                                & Resistor ID Module      \\ \hline
\textbf{M5}                                & Band Location Module    \\ \hline
\textbf{M6}                                & Colour Mapping Module   \\ \hline
\textbf{M7}                                & Value Calculator Module \\ \hline
\end{tabular}
\end{table}

\begin{table}[h!]
\centering
\begin{tabular}{p{0.3\textwidth} p{0.6\textwidth}}
\toprule
\textbf{Level 1} & \textbf{Level 2}\\
\midrule

{Hardware-Hiding Module} 
& Camera Input Module\\
& Image Input Module \\
\midrule

\multirow{1}{0.3\textwidth}{Behaviour-Hiding Module}
& User Interface Module\\
\midrule

\multirow{4}{0.3\textwidth}{Software Decision Module}
& Resistor ID Module \\
& Band Location Module \\
& Colour Mapping Module \\
& Value Calculator Module\\
\bottomrule

\end{tabular}
\caption{Module Hierarchy}
\label{TblMH}
\end{table}



\section{Connection Between Requirements and Design} \label{SecConnection}

The design of the system is intended to satisfy the requirements developed in
the SRS. In this stage, the system is decomposed into modules. The connection
between requirements and modules is listed in Table \ref{TblRT}.

\section{Module Decomposition} \label{SecMD}

The \emph{Secrets} field in a module decomposition is a brief statement of the design decision hidden by the module. 
The \emph{Services} field specifies \emph{what} the module will do without documenting \emph{how} to do it. 
For each module, a suggestion for the implementing software is given under the \emph{Implemented By} title. 
If the entry is \emph{OS}, this means that the module is provided by the operating system or by standard programming language libraries.
Also indicate if the module will be implemented specifically for the software.

Only the leaf modules in the hierarchy have to be implemented. 
If a dash (\emph{--}) is shown, this means that the module is not a leaf and will not have to be implemented. 
Whether or not this module is implemented depends on the programming language selected.

\subsection{Hardware Hiding Modules (\mref{mHH})}

\begin{description}
\item[Secrets:]The data structure and algorithm used to implement the virtual
  hardware.
\item[Services:]Serves as a virtual hardware used by the rest of the
  system. This module provides the interface between the hardware and the
  software. So, the system can use it to display outputs or to accept inputs.
\item[Implemented By:] OS
\end{description}

\subsection{Behaviour-Hiding Module}

\begin{description}
\item[Secrets:]The contents of the required behaviours.
\item[Services:]Includes programs that provide externally visible behaviour of
  the system as specified in the software requirements specification (SRS)
  documents. This module serves as a communication layer between the
  hardware-hiding module and the software decision module. The programs in this
  module will need to change if there are changes in the SRS.
\item[Implemented By:] --
\end{description}

\subsubsection{Input Format Module (\mref{mInput})}

\begin{description}
\item[Secrets:]The format and structure of the input data.
\item[Services:]Converts the input data into the data structure used by the
  input parameters module.
\item[Implemented By:] [Your Program Name Here]
\end{description}

\subsubsection{Etc.}


\subsection{Software Decision Module}

\begin{description}
\item[Secrets:] The design decision based on mathematical theorems, physical
  facts, or programming considerations. The secrets of this module are
  \emph{not} described in the SRS.
\item[Services:] Includes data structure and algorithms used in the system that
  do not provide direct interaction with the user. 
  % Changes in these modules are more likely to be motivated by a desire to
  % improve performance than by externally imposed changes.
\item[Implemented By:] --
\end{description}

\subsubsection{Etc.}

\section{Traceability Matrix} \label{SecTM}

This section shows two traceability matrices: between the modules and the
requirements and between the modules and the anticipated changes.

% the table should use mref, the requirements should be named, use something
% like fref
\begin{table}[H]
\centering
\begin{tabular}{p{0.2\textwidth} p{0.6\textwidth}}
\toprule
\textbf{Req.} & \textbf{Modules}\\
\midrule
R1 & \mref{mHH}, \mref{mInput}, \mref{mParams}, \mref{mControl}\\
R2 & \mref{mInput}, \mref{mParams}\\
R3 & \mref{mVerify}\\
R4 & \mref{mOutput}, \mref{mControl}\\
R5 & \mref{mOutput}, \mref{mODEs}, \mref{mControl}, \mref{mSeqDS}, \mref{mSolver}, \mref{mPlot}\\
R6 & \mref{mOutput}, \mref{mODEs}, \mref{mControl}, \mref{mSeqDS}, \mref{mSolver}, \mref{mPlot}\\
R7 & \mref{mOutput}, \mref{mEnergy}, \mref{mControl}, \mref{mSeqDS}, \mref{mPlot}\\
R8 & \mref{mOutput}, \mref{mEnergy}, \mref{mControl}, \mref{mSeqDS}, \mref{mPlot}\\
R9 & \mref{mVerifyOut}\\
R10 & \mref{mOutput}, \mref{mODEs}, \mref{mControl}\\
R11 & \mref{mOutput}, \mref{mODEs}, \mref{mEnergy}, \mref{mControl}\\
\bottomrule
\end{tabular}
\caption{Trace Between Requirements and Modules}
\label{TblRT}
\end{table}

\begin{table}[H]
\centering
\begin{tabular}{p{0.2\textwidth} p{0.6\textwidth}}
\toprule
\textbf{AC} & \textbf{Modules}\\
\midrule
\acref{acHardware} & \mref{mHH}\\
\acref{acInput} & \mref{mInput}\\
\acref{acParams} & \mref{mParams}\\
\acref{acVerify} & \mref{mVerify}\\
\acref{acOutput} & \mref{mOutput}\\
\acref{acVerifyOut} & \mref{mVerifyOut}\\
\acref{acODEs} & \mref{mODEs}\\
\acref{acEnergy} & \mref{mEnergy}\\
\acref{acControl} & \mref{mControl}\\
\acref{acSeqDS} & \mref{mSeqDS}\\
\acref{acSolver} & \mref{mSolver}\\
\acref{acPlot} & \mref{mPlot}\\
\bottomrule
\end{tabular}
\caption{Trace Between Anticipated Changes and Modules}
\label{TblACT}
\end{table}

\section{Use Hierarchy Between Modules} \label{SecUse}

In this section, the uses hierarchy between modules is
provided. \citet{Parnas1978} said of two programs A and B that A {\em uses} B if
correct execution of B may be necessary for A to complete the task described in
its specification. That is, A {\em uses} B if there exist situations in which
the correct functioning of A depends upon the availability of a correct
implementation of B.  Figure \ref{FigUH} illustrates the use relation between
the modules. It can be seen that the graph is a directed acyclic graph
(DAG). Each level of the hierarchy offers a testable and usable subset of the
system, and modules in the higher level of the hierarchy are essentially simpler
because they use modules from the lower levels.

\begin{figure}[H]
\centering
%\includegraphics[width=0.7\textwidth]{UsesHierarchy.png}
\caption{Use hierarchy among modules}
\label{FigUH}
\end{figure}

%\section*{References}

\bibliographystyle {plainnat}
\bibliography {MG}

\end{document}